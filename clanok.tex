\documentclass[10pt,twoside,slovak,a4paper]{article}

\usepackage[slovak]{babel}
\usepackage[T1]{fontenc}
\usepackage[IL2]{fontenc} 
\usepackage[utf8]{inputenc}
\usepackage{graphicx}
\usepackage{url}
\usepackage{hyperref} 
\usepackage{cite}
\usepackage{times}

\pagestyle{headings}

\title{Gamifikácia v marketingu\thanks{Semestrálny projekt v predmete Metódy inžinierskej práce, ak. rok 2022/23, vedenie: Zuzana Špitálová}} 

\author{Maroš Kožár\\[2pt]
	{\small Slovenská technická univerzita v Bratislave}\\
	{\small Fakulta informatiky a informačných technológií}\\
	{\small \texttt{xkozar@stuba.sk}}
	}
\date{\small 16. november 2022}
\begin{document}
\maketitle
\section{Úvod}
V posledných rokoch získala "gamifikácia" významnú pozornosť odborníkov z praxe a herných expertov.
Avšak súčasné chápanie gamifikácie je založené výlučne na pridávaní herných prvkov do služieb marketingu.
V tomto článku navrhujeme novú definíciu gamifikácie, ktorá zdôrazňuje zážitkovú povahu hier, namiesto systémového chápania.
Základný problém, ktorý bol naznačený v úvode, je podrobnejšie vysvetlený v časti marketingové služby.
Dôležité súvislosti sú uvedené v častiach hry ako systémy služieb a systémy služieb.
Záverečné poznámky prináša časť záver.



\section{Marketingové služby} \label{nejaka}


\begin{figure*}[tbh]
Dva kľúčové koncepty prístupu k službám, zákazník ako spolutvorca a výmena hodnoty, pomáhajú vysvetliť všadeprítomný rozdiel medzi tradičnou produktovo zameranou logikou a novou  logikou zameranou na služby.
V tradičnej marketingovej teórii sa výroba považuje za zodpovednosť spoločnosti
a hodnota produktu je vytváraná vo výrobnom procese a zakomponovaná vo výslednom výrobku. Výrobok potom "nesie"
hodnotu a hodnota sa prenáša z podniku na
zákazníka pomocou platby. V kontexte služieb sa však tento
prístup založený na výmene hodnoty stráca zmysel, pretože neexistuje
fyzický výrobok, ku ktorému by sa mohla priradiť hodnota.
\end{figure*}



\section{Pravidlá hry} \label{ina}

Pojem "gamifikácia" bol prvýkrát použitý v roku 2008 v príspevku na blogu
Brett Terill. On ako prvý opísal termín vybratia hernej mechaniky a použitia jej v iných odvetviach na zvýšenie efektivity služieb. Tento termín sa neskôr zjednodušil do pojmu gamifikácia ktorý je omnoho viac používaný. Napriek vzniku nápadu už v roku 2008, odvetvia reagujú a menia sa veľmi pomaly. Definícia gamifikácie sa časom mení a teraz opisuje použitie dizajnových prvkov hier v nehernom kontexte, 







\section{Služby a systém služieb} \label{dolezita}
Na účely definovania gamifikácie sa používajú tri kľúčové pojmy
marketingu služieb: služba, systém služieb a
balík služieb.
Vargo a Lusch definujú služby ako aplikácie špecializovaných zručností (vedomosti a zrunčnosti), cez skutky procesy a výkony kvôli výhodám tretej osoby alebo samých seba. Takže, každý pripravený krok, nezáleží ako malý ktorý pomôže osobe je považovaný za službu. Systematický balík služieb tvorí systém ktorý je zložený z častí zdrojov vrátane ľudí technológií informácií atď. pripojený ku iným systémom podľa ponákanej hodnoty. Cieľom systému služieb je použitie jeho zdrojov a zdrojov iných na vylepšenie okolností svojich tak ako aj iných. 



\section{Hry ako systémy služieb} \label{dolezitejsia}
Videné prostredníctvom literatúry marketingu služieb, herné dizajnové elementy môžu byt opísané ako služby a hry ako systémy služieb. Toto je podporované faktom, že hry sú vždy považované sa systémy ktoré vyžadujú aktívnu účasť hráča. Hry sú preto spoluvytvorené herným vývojárom a hráčom. Časť herného vývojára je vytvoriť príbeh hry, pravidlá hry, herné prvky, vizuálne efekty atď. Časť hráča je vytváranie hodnoty. To je robené vždy, keď hráč komunikuje s hrou, hrá ju. Hra môže byť taktiež čiastočne alebo úplne vytvorená hráčom. Hlavnou úlohou hry je poskytnúť hráčovi čo najlepší zážitok.


\section{Záver} \label{zaver} % prípadne iný variant názvu
Jedným zo zaujímavých smerov budúceho výskumu by mohlo byť skúmanie
zákazníckych vernostných kariet a iných široko používaných marketingových techník
ako herné služby. Gamifikácia by sa mohla použiť aj
na rozšírenie obzorov služieb, ktorý predstavil Bitner v roku 1992,
z fyzických nastavení na abstraktnejšie konštrukcie. Obzor služieb nám dáva predstavu, ktorým smerom sa služby môžu vydať. Tento obzor ovplyvňuje zákazníkove správanie a očakávania. Príkladom by mohol byť obchod IKEA. Rozloženie tovaru v obchode núti zákazníka ísť určitou cestou a ovplyvňuje čo si kúpi, gamifikácia by mohla byť využitá na vylepšenie takýchto situácií, a tiež by mohla poskytnúť zákazníkom viac potenciálnych ciest ktorými sa budú chcieť vydať.


% týmto sa generuje zoznam literatúry z obsahu súboru literatura.bib podľa toho, na čo sa v článku odkazujete
\bibliography{literatura}
\bibliographystyle{plain} % prípadne alpha, abbrv alebo hociktorý iný
\end{document}
