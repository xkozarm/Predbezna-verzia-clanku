\documentclass[10pt,twoside,slovak,a4paper]{article}

\usepackage[slovak]{babel}
\usepackage[T1]{fontenc}
\usepackage[IL2]{fontenc} 
\usepackage[utf8]{inputenc}
\usepackage{graphicx}
\usepackage{url}
\usepackage{hyperref} 
\usepackage{cite}
\usepackage{times}

\pagestyle{headings}

\title{Gamifikácia v marketingu\thanks{Semestrálny projekt v predmete Metódy inžinierskej práce, ak. rok 2022/23, vedenie: Zuzana Špitálová}} 

\author{Maroš Kožár\\[2pt]
	{\small Slovenská technická univerzita v Bratislave}\\
	{\small Fakulta informatiky a informačných technológií}\\
	{\small \texttt{xkozar@stuba.sk}}
	}
\date{\small 16. november 2022}
\begin{document}
\maketitle
\section{Úvod}
V posledných rokoch získala "gamifikácia" významnú pozornosť odborníkov z praxe a herných expertov.
Avšak súčasné chápanie gamifikácie bolo založené výlučne na akte pridávania systémových herných prvkov do služieb.
V tomto článku navrhujeme novú definíciu gamifikácie, ktorá zdôrazňuje zážitkovú povahu hier a gamifikácie, namiesto systémového chápania.
Základný problém, ktorý bol naznačený v úvode, je podrobnejšie vysvetlený v časti marketingové služby~\ref{nejaka}.
Dôležité súvislosti sú uvedené v častiach hry ako systémy služieb a systémy služieb~\ref{dolezita} a~\ref{dolezitejsia}.
Záverečné poznámky prináša časť záver~\ref{zaver}.



\section{Marketingové služby} \label{nejaka}


\begin{figure*}[tbh]
Dva kľúčové koncepty prístupu k službám, zákazník ako spolutvorca a hodnota pri používaní, pomáhajú vysvetliť všadeprítomnú
a hlboký rozdiel medzi logikou služieb a logikou služieb.
medzi tradičnou logikou, v ktorej dominuje tovar, a novou logikou
logikou dominantnej služby.
V tradičnej marketingovej teórii sa výroba považuje za
a hodnota sa považuje za vytvorenú
počas výrobného procesu spoločnosťou a že je
zakotvená vo výslednom produkte. Výrobok potom "nesie"
hodnotu v ňom a hodnota sa prenáša z podniku na
zákazníka spolu s transakciou. V kontexte služieb sa však tento
prístup založený na výmene hodnoty stráca zmysel, pretože neexistuje
fyzický výrobok, ku ktorému by sa mohla priradiť hodnota.
\end{figure*}



\section{Pravidlá hry} \label{ina}

Pojem "gamifikácia" bol prvýkrát použitý v roku 2008 v príspevku na blogu
Brett Terill [29]. Tento termín opísal ako "prevzatie hry
mechaniky a ich použitie na iné webové vlastnosti s cieľom zvýšiť
angažovanosť. K širšiemu používaniu v odvetví sa termín dostal
počas roka 2010 vo svojej súčasnej podobe "gamifikácia" [7].
Napriek pozornosti, ktorú si termín v odvetví rýchlo získal,
akademická obec reagovala pomaly. Podľa našich vedomostí existujú
len dve definície gamifikácie: definícia, ktorú uviedol Deterding
et al. [7] a tá, ktorá bola uvedená v prvej skrátenej verzii a teraz
drasticky odlišnej verzii tohto dokumentu. 







\section{Služby a systém služieb} \label{dolezita}
Na účely definovania gamifikácie sa používajú tri kľúčové pojmy
marketingu služieb: služba, systém služieb a
balík služieb.
Vargo a Lusch [31] definujú službu ako "aplikáciu
špecializovaných kompetencií (vedomostí a zručností) prostredníctvom skutkov,
procesov a výkonov v prospech iného subjektu alebo
samotného subjektu". Teda každý úmyselný čin - bez ohľadu na to, ako malý -
ktorý pomáha nejakému subjektu, možno považovať za službu.
Systematický súbor služieb predstavuje systém služieb, ktorý,
podľa [25] "predstavuje usporiadanie zdrojov (vrátane
ľudí, technológií, informácií atď.) prepojených s inými systémami
hodnotovými ponukami". Cieľom systému služieb je využiť jeho
zdroje a zdroje iných subjektov na zlepšenie svojej situácie
a ostatných [33]. 
Premýšľanie o tom, čo je "plnohodnotná hra" a čo nie, bude len
dizajnérov odvádza od toho, na čo by sa mali zamerať:
zákazník/užívateľ/hráč.
Tieto nezrovnalosti nás viedli k hľadaniu alternatívneho spôsobu, ako
definovať gamifikáciu z pohľadu marketingu služieb



\section{Hry ako systémy služieb} \label{dolezitejsia}
 Videné prostredníctvom literatúry o marketingu služieb,
prvky herného dizajnu možno opísať ako služby a hry ako
systémy služieb. Potvrdzuje to tabuľka 1, ktorá ukazuje, že
hry sa vždy považujú za systémy, ktoré si vyžadujú aktívnu
zapojenie hráča.
Hry sú teda spoluvytvárané tvorcom hry a
hráčom (hráčmi). Časť koprodukcie zo strany vývojára hry zaberá
keď sa vytvára príbeh hry, vymýšľajú sa pravidlá
vyberajú sa vzory herného dizajnu a navrhuje sa vizuálna stránka atď. Hráč (hráči)
sa na koprodukcii a tvorbe hodnôt podieľa hráč
pri každom hraní hry alebo inej interakcii s ňou. Stránka
môže byť tiež výlučne alebo čiastočne vytvorená hráčom.
samozrejme. Základnou službou hry je poskytovať hedonické,
náročné a napínavé zážitky pre hráča (hráčov) [21] alebo
herné zážitky [22]. Kvalita takejto "hernej služby" je
výrazne determinovaná funkčnou kvalitou služby, resp.
herného zážitku, ktorá sa často označuje ako flow [6]. 



\section{Záver} \label{zaver} % prípadne iný variant názvu
Jedným zo zaujímavých smerov budúceho výskumu by mohlo byť skúmanie
zákazníckych vernostných kariet a iných široko používaných marketingových
ako herné služby. Gamifikácia by sa mohla použiť aj
na rozšírenie modelu krajiny služieb, ktorý predstavil Bitner v roku 1992,
z fyzických prostredí na abstraktnejšie konštrukcie, ako to urobili [1].
navrhli. Krajina služieb poskytuje rámec pre krajinu
kde sa služba uskutočňuje a ktorá je pod kontrolou



% týmto sa generuje zoznam literatúry z obsahu súboru literatura.bib podľa toho, na čo sa v článku odkazujete
\bibliography{literatura}
\bibliographystyle{plain} % prípadne alpha, abbrv alebo hociktorý iný
\end{document}
